\documentclass[12pt]{article}
\usepackage{pmmeta}
\pmcanonicalname{PoliteNumber}
\pmcreated{2013-03-22 18:09:54}
\pmmodified{2020-02-25 18:09:54}
\pmowner{PrimeFan}{13766}
\pmmodifier{PrimeFan}{13766}
\pmformalizer{ngonga}{}
\pmtitle{polite number}
\pmrecord{6}{40725}
\pmprivacy{1}
\pmauthor{PrimeFan}{13766}
\pmtype{Definition}
\pmcomment{trigger rebuild}
\pmclassification{msc}{11A25}


\defines{polite number}

\endmetadata

% this is the default PlanetMath preamble.  as your knowledge
% of TeX increases, you will probably want to edit this, but
% it should be fine as is for beginners.

% almost certainly you want these
\usepackage{amssymb}
\usepackage{amsmath}
\usepackage{amsfonts}

\usepackage{cnl}
\usepackage{xcolor}


% TITLE AUTHOR DATE
\title{Number Theory,\\ PoliteNumber}
\date{February 25, 2020}
\author{Ngo Nga (ngongatlu)}



%- DOCUMENT

\begin{document}
\parskip=\baselineskip

% Local Defs 
\parindent=0pt

\begin{cnl}

\Cnlinput{../TeX2CNL/package/cnlinit}

\bigskip

%-% BEGIN


\lsection{}
In this section, let $d,\ m,\ k ,\ a$ be integers.

\dfn{
We say that $d$ is \df{positive} iff $d > 0$.
}

\dfn{We say $a$ is a \df{polite\~number} iff there exist positive integer $k$ and positive integer $m$ such that $k\geq 2$
 and $2\*a=k\*(2\*m+k-1)$. 
}

\begin{demark}
A {\em polite number} $n$ is an integer that is the sum of two or more consecutive nonnegative integers in 
at least one way.

 To put it algebraically, if $n$ is polite then there is a solution to $$n = \sum_{i = a}^b i$$ with 
$b > a$ and $a > -1$. For example, 42 is a polite number since it is the sum of the integers from 3 to 9. 
The first few polite numbers are 3, 5, 6, 7, 9, 10, 11, 12, 13, 14, 15, 17, 18, 19, 20, 21, 22, 23, 24, 
25, 26, 27, 28, 29, 30, 31, 33, 34, 35, 36, 37, 38, 39, 40, etc.

Obviously all triangular numbers are polite numbers. So are all odd numbers. In fact, the numbers that are
 not polite are the powers of 2.
\end{demark}



\end{cnl}
%%%%%
%%%%%
\end{document}
