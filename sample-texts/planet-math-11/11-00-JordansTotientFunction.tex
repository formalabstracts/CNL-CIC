\documentclass[12pt]{article}
\usepackage{pmmeta}
\pmcanonicalname{JordansTotientFunction}
\pmcreated{}
\pmmodified{}
\pmmodifier{}{}
\pmowner{mathcam}{2727}
\pmformalizer{miinguyen}{}  % Note this new field
\pmtitle{Jordan's totient function}
\pmrecord{39}{30010}
\pmprivacy{1}
\pmauthor{miinguyen}{}
\pmtype{Definition}
\pmcomment{trigger rebuild}
\pmclassification{msc}{11-00}
\pmclassification{msc}{46M15}
\pmclassification{msc}{18C15}

\endmetadata


\usepackage{amssymb}
\usepackage{amsmath}
\usepackage{amsfonts}
\usepackage{graphicx}

\usepackage{cnl}
\usepackage{xcolor}

% TITLE AUTHOR DATE
\title{Number Theory,\\ Jordan's totient function}
\date{March 05, 2019}
\author{My Nguyen (miinguyen)}



\begin{document}
\parskip=\baselineskip

% Local Defs for Coprime definition 
\def\natdiv#1#2{{#1}\mathrel{|}{#2}}
\newcommand{\intpow}[2]{{#1}^{#2}}

\begin{cnl}
\Cnlinput{../TeX2CNL/package/cnlinit}

\bigskip


In this section, let $d,\ m$ be integers.

\dfn{We say that $d$ \df{divides} $m$ iff $d\ne 0$ and there exists an integer $r$ such that $m= d\*r$.}
	
\dfn{
We write  $\natdiv{d}{m}$ iff $d$ divides $m$.	
}

\dfn{
We say  $d$ is a \df{divisor} of $m$ iff $d$ divides $m$. 
}

\dfn{
Assume that $p$ is an integer greater than $1$.  Then we say that $p$
is a \df{prime} iff each divisor of $p$ is either equal to $1$ or equal to $p$.
}


\dfn{
Assume that $p$ is an integer greater than $1$.  Then we say that $p$
is a \df{prime divisor} of $n$ iff $p$ is a prime and $p$ is a divisor of $n$.
}

\dfn{Assume $k$ and $n$ are natural numbers. Then \df{Jordan's totient function} $J_k(n)$ is equal to $n^k\prod_{\natdiv{p}{n}} ( 1-p^{-k})$ where the product is over prime divisor $p$ of $n$.}

\end{cnl}
%%%%%
%%%%%
\end{document}
