\documentclass[12pt]{article}
\usepackage{pmmeta}
\pmcanonicalname{JordansTotientFunction}
\pmcreated{}
\pmmodified{}
\pmmodifier{}{}
\pmowner{mathcam}{2727}
\pmformalizer{miinguyen}{}  % Note this new field
\pmtitle{Jordan's totient function}
\pmrecord{39}{30010}
\pmprivacy{1}
\pmauthor{miinguyen}{}
\pmtype{Definition}
\pmcomment{trigger rebuild}
\pmclassification{msc}{11-00}
\pmclassification{msc}{46M15}
\pmclassification{msc}{18C15}

\endmetadata


\usepackage{amssymb}
\usepackage{amsmath}
\usepackage{amsfonts}
\usepackage{graphicx}

\usepackage{cnl}
\usepackage{xcolor}

% TITLE AUTHOR DATE
\title{Number Theory,\\ Jordan's totient function}
\date{March 05, 2019}
\author{My Nguyen (miinguyen)}



\begin{document}
\parskip=\baselineskip

% Local Defs for Coprime definition 
\def\natdiv#1#2{{#1}\mathrel{|}{#2}}
\def\natpow#1#2{{#1}^{#2}}

\begin{cnl}
\Cnlinput{../TeX2CNL/package/cnlinit}

\bigskip

\lsection{Prime divisor number}

In this section, let $d,\ m$ be integers.

\dfn{
Assume that $m,\ d$ are integers. Then we say that $d$ \df{divides} $m$ iff $d\ne 0$ and there exists an integer $r$ such that $m= d\*r$.
}
	
\dfn{
We write  $\natdiv{d}{m}$ iff $d$ divides $m$.	
}

\dfn{
We say  $d$ is a \df{divisor} of $m$ iff $d$ divides $m$. 
}

\dfn{
Assume that $p$ is an integer greater than $1$.  Then we say that $p$
is a \df{prime} iff each divisor of $p$ is either equal to $1$ or equal to $p$.
}


\dfn{
Assume that $p$ is an integer greater than $1$.  Then we say that $p$
is a \df{prime divisor} of $n$ iff $p$ is a prime and $p$ is a divisor of $n$.
}


\lsection{List}
In this section, let $\alpha$ be a type.

%def of list of type alpha
\dfn{
Let \df{list} of $alpha$ be the inductive type
\begin{envMatch}
\firstmatchitem $\nullbrack$ &$:$& $\h{list}$
\matchitem $\h{cons}$ &$:$& $\alpha\to\h{list}\to\h{list}$\texstop
\end{envMatch}\cnlstop
}


%def of function append
\dfn{
Let \df{prime-div-list} $(n : \Nat)$ $: \Nat \to \h{list}\ \alpha\ \assign$ \par\function
\begin{envMatch}
\firstmatchitem $0$ &$\assign$& $\nullbrack$
\matchitem $(succ\ p)$ &$\assign$& if $\natdiv{p}{n}$ and $p$ is a prime and $p \leq n$ then $p \cons$ \h{prime-div-list} $n$\texstop
\end{envMatch}\cnlstop
This exists by recursion.
}


\lsection{Jordan's totient function}
In this section, let $\alpha$ be a type.
Let $R$ be type of real number. %need concept of real number


%give definition of product
\dfn{
Let \df{prod-Jordan-func}\ $: \h{list}\ \alpha\ \to R \assign$ \par\function
\begin{envMatch}
\firstmatchitem $\nullbrack$ &$\assign$& $1$
\matchitem $a \cons A$ &$\assign$& $(1-\natpow{a}{-k}) \* \h{prod-Jordan-func}\ A$\texstop
\end{envMatch}\cnlstop
This exists by recursion.
}



\dfn{Assume $k$ and $n$ are natural numbers. Then \df{Jordan's totient function} $J_k(n)$ is equal to $\natpow{n}{k} \* \h{prod-Jordan-func}\ (\h{prime-div-list}\ n)$ where the product is over prime divisor $p$ of $n$.}

\begin{remark}
   The explicite formula of Jordan's totient function is 
    $$J_k(n) = n^k \prod_{p|n} ( 1-p^{-k})$$
    where the product is over prime divisor $p$ of $n$.
\end{remark}

\end{cnl}
%%%%%
%%%%%
\end{document}
