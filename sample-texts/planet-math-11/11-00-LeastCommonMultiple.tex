\documentclass[12pt]{article}
\usepackage{pmmeta}
\pmcanonicalname{LeastCommonMultiple}
\pmcreated{2015-05-06 19:07:25}
\pmmodified{2020-02-09}
\pmowner{pahio}{2872}
\pmmodifier{pahio}{2872}
\pmformalizer{ngonga}{}
\pmtitle{least common multiple}
\pmrecord{32}{35723}
\pmprivacy{1}
\pmauthor{pahio}{2872}
\pmtype{Definition}
\pmcomment{trigger rebuild}
\pmclassification{msc}{11-00}
\pmsynonym{least common dividend}{LeastCommonMultiple}
\pmsynonym{lcm}{LeastCommonMultiple}
\pmrelated{Divisibility}
\pmrelated{PruferRing}
\pmrelated{SumOfIdeals}
\pmrelated{IdealOfElementsWithFiniteOrder}

\pmdefines{divides}
\pmdefines{multiple}
\pmdefines{common multiple}
\pmdefines{least common multiple}


\endmetadata


\usepackage{amssymb}
\usepackage{amsmath}
\usepackage{amsfonts}

\usepackage{cnl}
\usepackage{xcolor}

\DeclareMathOperator{\lcm}{lcm}

% TITLE AUTHOR DATE
\title{Number Theory,\\ LeastCommonMultiple}
\date{February 09, 2020}
\author{Ngo Nga (ngongatlu)}


%- DOCUMENT

\begin{document}
\parskip=\baselineskip

% Local Defs 
\def\natdiv#1#2{{#1}\mathrel{|}{#2}}
\parindent=0pt

\begin{cnl}

\Cnlinput{../TeX2CNL/package/cnlinit}

\bigskip

%-% BEGIN

\lsection{LeastCommonMultiple}
In this section, let $m,\ d$ be integers.

\deflabel{divides}
Assume that $m, \d$ are integers. Then we say that $d$ \df{divides} $m$ iff $d\ne 0$ and there 
exists an integer $r$ such that $m= d\*r$.
\end{definition}

\dfn{
We write $\natdiv{d}{m}$ iff $d$ divides $m$.
}

\dfn{
We say that $m$ is a \df{mutiple of} $d$ iff $d$ divides $m$.
}

\dfn{
We say that $d$ is \df{positive} iff $d > 0$.
}

\dfn{
Assume that $f,\ p,\ q$ are integers. Then we say $f$ is a \df{common\~multiple} of $p$ and $q$ 
iff $p$ divides $f$ and $q$ divides $f$.}


\begin{remark}
We now introduce the least common multiple: 
\end{remark}

\dfn{Assume that $p,\ q, \ f$ are positive integers. Then we say that $f$ is a 
\df{least\~common\~multiple} of $p$ and $q$  iff $f$ is a positive common\~multiple of $p$ and
 $q$ such that for every common\~multiple $r$ of $p$ and $q$ we have $\natdiv{f}{r}$.}

\dfn{ Assume that  $p,\ q$ are positive integers. Let $\h{\df{lcm}}\ p\ q$ denote the 
least\~common\~multiple of $p$ and $q$. This exists and is unique.}




\begin{demark}
\textbf{Note:} \, The definition can be generalized for several 
numbers. \,The positive {lcm of positive integers is 
uniquely determined. (Its negative satisfies the same two 
conditions.)

\subsection*{Properties}

\begin{enumerate} 
  \item If\, $a = \prod_{i=1}^{m}p_i^{\alpha_i}$\, and\, 
  $b = \prod_{i=1}^{m}p_i^{\beta_i}$\, are the prime factor 
  \PMlinkescapetext{presentations} of the positive integers $a$ and $b$ ($\alpha_{i} \geqq 0$, \,$\beta_{i} \geqq 0$ \,$\forall i$), then 
        $$\mathrm{lcm}\!(a,\,b)= 
        \prod_{i=1}^{m}p_i^{\max\{\alpha_i,\,\beta_i\}}.$$ 
This can be generalized for lcm of several numbers.
  \item  Because the greatest common divisor has the expression\, 
  $\gcd(a,\,b) = \prod_{i=1}^{m}p_i^{\min\{\alpha_i,\,\beta_i\}}$, we see that  
  $$\gcd(a,\,b)\cdot \mathrm{lcm}\!(a,\,b) = ab.$$
This formula is sensible only for two integers; it can not be 
generalized for several numbers, i.e., for example,
       $$\gcd(a,\,b,\,c)\cdot \mathrm{lcm}(a,\,b,\,c) \neq abc.$$
  \item The preceding formula may be presented in 
  \PMlinkescapetext{terms} of ideals of $\mathbb{Z}$; we may 
  replace the integers with the corresponding principal ideals.\, 
  The formula acquires the form
      $$((a)+(b))((a)\cap(b)) = (a)(b).$$
  \item The recent formula is valid also for other than principal ideals and even in so general systems as the Pr\"ufer rings; in fact, it could be taken as defining property of these rings: \, Let $R$ be a commutative ring with non-zero unity. \,$R$ is a Pr\"ufer ring iff {\em Jensen's formula}
 $$(\mathfrak{a}+\mathfrak{b})(\mathfrak{a}\cap\mathfrak{b}) = \mathfrak{ab}$$
is true for all ideals $\mathfrak{a}$ and $\mathfrak{b}$ of $R$, with at least one of them having \PMlinkname{non-zero-divisors}{ZeroDivisor}.
\end{enumerate}

\end{demark}

\begin{thebibliography}{9}
\bibitem{Larsen & McCarthy} M. Larsen and P. McCarthy: {\em Multiplicative theory of ideals}. Academic Press. New York (1971).
\end{thebibliography}
%%%%%
%%%%%

\end{cnl}
\end{document}
