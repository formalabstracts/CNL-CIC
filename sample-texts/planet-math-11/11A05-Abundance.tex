\documentclass[12pt]{article}
\usepackage{pmmeta}
\pmcanonicalname{Abundance}
\pmcreated{}
\pmmodified{}
\pmowner{}{}
\pmmodifier{}{}
\pmtitle{abundance}
\pmrecord{9}{38159}
\pmprivacy{1}
\pmauthor{miinguyen}{}
\pmtype{Definition}
\pmcomment{trigger rebuild}
\pmclassification{msc}{11A05}
\pmrelated{Deficiency}

\endmetadata


\usepackage{amssymb}
\usepackage{amsmath}
\usepackage{amsfonts}
\usepackage{graphicx}

\usepackage{cnl}
\usepackage{xcolor}

% TITLE AUTHOR DATE
\title{Number Theory,\\ Abundance}
\date{March 15, 2020}
\author{My Nguyen (miinguyen)}

\begin{document}
\parskip=\baselineskip

% Local Defs for Coprime definition 
\def\natdiv#1#2{{#1}\mathrel{|}{#2}}
\def\natpow#1#2{{#1}^{#2}}

\begin{cnl}
\Cnlinput{../TeX2CNL/package/cnlinit}

\bigskip

\lsection{Divisor}

In this section, let $d,\ m$ be integers.

\dfn{
Assume that $m,\ d$ are integers. Then we say that $d$ \df{divides} $m$ iff $d\ne 0$ and there exists an integer $r$ such that $m= d\*r$.
}
	
\dfn{
We write  $\natdiv{d}{m}$ iff $d$ divides $m$.	
}

\dfn{
We say  $d$ is a \df{divisor} of $m$ iff $d$ divides $m$. 
}

\begin{remark}
Function to return $k$ if $k$ is a divisor of $n$. It is intended mainly for implementing in the function 'sum-f' below.
\end{remark}


\dfn{
Let
$\df{divisor-of-n}\ (n\ : \Nat)\assign$
\begin{align*}
\funalign
{\Nat}&{\quad\to \ }{\Nat}\\
{k}&{\quad\mapsto\ }
{
\begin{cases}
\firstmatchitem \caseif{k}{\natdiv{k}{n} }
\matchitem \caseotherwise{0}\texstop
\end{cases}
}
\end{align*}
\cnlstop
}


\lsection{List}
In this section, let $\alpha$ be a $\Type$.

\begin{remark}
Definition of a list of type $\alpha$.
\end{remark}

\dfn{
Let \df{list} of $alpha$ be the inductive type
\begin{envMatch}
\firstmatchitem $\nullbrack$ &$:$& $\h{list}$
\matchitem $\h{cons}$ &$:$& $\alpha\to\h{list}\to\h{list}$\texstop
\end{envMatch}\cnlstop
}

%Sum all elements of the list.
\begin{remark}
    Function return sum of all elements of a list.
\end{remark}

\dfn{
Let \df{sum} $: \h{list}\ \alpha\ \to \Nat \assign$ \par\function
\begin{envMatch}
\firstmatchitem $\nullbrack$ &$\assign$& $0$
\matchitem $a \cons A$ &$\assign$& $a + \h{sum} A$\texstop
\end{envMatch}\cnlstop
This exists by recursion.
}


\begin{remark}
    Function return the sum with function apply to elements of the list. This function allow us to take the sum in general form $\left( \sum_{i = 1}^n f(i) \right)$
\end{remark}

\dfn{
Let \df{sum-f} $(f\ :\ \alpha\ \to \Nat)$ $: \h{list}\ \alpha\ \to \Nat \assign$ \par\function
\begin{envMatch}
\firstmatchitem $\nullbrack$ &$\assign$& $0$
\matchitem $a \cons A$ &$\assign$& $f(a) + \h{sum-f}\ A$\texstop
\end{envMatch}\cnlstop
This exists by recursion.
}


\begin{remark}
    Function 'range s t' return the list of numbers [s, s+1, s + t - 1]. It is intended for taking sum in a range.
\end{remark}

\dfn{
Let \df{range} $: \Nat \to \Nat \to \h{list}\ \Nat \assign$ \par\function
\begin{envMatch}
\firstmatchitem $s\ 0$ &$\assign$& $\nullbrack$
\matchitem $s\ (t+1)$ &$\assign$& $s \cons \h{range}\ (s+1)\ t $\texstop
\end{envMatch}\cnlstop
This exists by recursion.
}


\lsection{Abundance}

\dfn{
    Let $n$ be an integer. The \df{abundance} of $n$ is the difference $$\h{sum-f}\ (\h{divisor-of-n}\ n)\ (\h{range}\ 1\ n) - 2 \* n$$ or  $$\h{sum-f}\ (\h{divisor-of-n}\ n)\ (\h{range}\ 1\ (n-1)) - n$$.
}


\begin{remark}
Given an integer $n$ with divisors $d_1, \ldots , d_k$ (where the divisors are in ascending order and $d_1 = 1$, $d_k = n$) the difference $$\left( \sum_{i = 1}^k d_i \right) - 2n$$ is the {\em abundance} of $n$. Or if one prefers, $$\left( \sum_{i = 1}^{k - 1} d_i \right) - n.$$

For example, the divisors of 12 (which are 1, 2, 3, 4, 6 and 12) add up to 28, which is 4 more than 24 (twice 12). Therefore, 12 has an abundance of 4. For the sake of comparison, the divisors of 13 are 1 and 13, adding up to 14, which is 12 less than 26 (twice 13). Therefore, 13 has an abundance of $-12$. A033880 in Sloane's OEIS lists the abundance of the first sixty-three positive integers.

\end{remark}


\dfn{The \df{abundant\~numbers} are the numbers with positive abundance.}

\dfn{The \df{quasiperfect\~number} is a number with an abundance of $1$.}

\dfn{The \df{perfect\~number} is a number with an abundance of $0$.}

\dfn{The \df{almost\~perfect\~number} is a number with an abundance of $-1$.}

\dfn{The \df{deficient\~numbers} are the numbers with negative abundance.}


\end{cnl}
%%%%%
%%%%%
\end{document}