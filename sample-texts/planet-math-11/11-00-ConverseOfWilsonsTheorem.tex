\documentclass[12pt]{article}
\usepackage{pmmeta}
\pmcanonicalname{ConverseOfWilsonsTheorem}
\pmcreated{2013-03-22 17:58:55}
\pmmodified{2020-01-09}
\pmowner{PrimeFan}{13766}
\pmmodifier{PrimeFan}{13766}
\pmformalizer{ngonga}{}
\pmtitle{converse of Wilson's theorem}
\pmrecord{7}{40491}
\pmprivacy{1}
\pmauthor{PrimeFan}{13766}
\pmtype{Theorem}
\pmcomment{trigger rebuild}
\pmclassification{msc}{11-00}

\pmdefines{divides}
\pmdefines{divisor}
\pmdefines{prime}
\pmdefines{factorial}

\endmetadata


\usepackage{amssymb}
\usepackage{amsmath}
\usepackage{amsfonts}

\usepackage{cnl}
\usepackage{xcolor}


% TITLE AUTHOR DATE
\title{Number Theory,\\ ConverseOfWilsonsTheorem}
\date{January 09, 2020}
\author{Ngo Nga (ngongatlu)}



%- DOCUMENT

\begin{document}
\parskip=\baselineskip

% Local Defs 
\def\natdiv#1#2{{#1}\mathrel{|}{#2}}
\def\factorial#1{{#1!}}
\parindent=0pt


\begin{cnl}

\Cnlinput{../TeX2CNL/package/cnlinit}

\bigskip

%-% BEGIN

\lsection{ConverseOfWilsonsTheorem}
In this section, let $m,\ d$ be integers.


\deflabel{divides}
Assume that $m, \d$ are integers. Then we say that $d$ \df{divides} $m$ iff $d\ne 0$ and there exists an integer $r$ such that $m= d\*r$.
\end{definition}

\dfn{
We write $\natdiv{d}{m}$ iff $d$ divides $m$.
}

\dfn{
We say that $d$ is a \df{divisor of} $m$ iff $d$ divides $m$.
}

\dfn{
We say that $d$ is \df{positive} iff $d > 0$.
}

\deflabel{prime}
Assume that $p$ is a natural number greater than $1$.  Then we say that $p$ is a \df{prime} iff each positive divisor of $p$ is equal to $1$ or equal to $p$.
\end{definition}

\deflabel{factorial}
Assume that $n$ are natural numbers.  Then we define
$\factorial{n}\assign$
\par$\match\ n$ with
\begin{envMatch}
\firstmatchitem $0$ &$\assign$& $1$
\matchitem $m + 1$ &$\assign$& $(m+1) \* \factorial{m}$\texstop
\end{envMatch}\cnlstop
This exists by recursion.
\end{definition}


\begin{theorem}[WilsonsTheorem]
Assume that $n$ is an integer. If $n>1$ and $n$ is a divisor of $\factorial(n - 1)+1$ then $n$ is prime.
\end{theorem}

\begin{demark}
To prove the converse of Wilson's theorem it is enough to show that a composite number can't satisfy the congruence. A number that does satisfy the congruence, then, would be not composite, and therefore prime.

\begin{proof}
If $n$ is composite, then its greatest prime factor is at most $\displaystyle \frac{n}{2}$, and $\displaystyle \frac{n}{2} < (n - 1)$ as long as $n > 2$ (and the smallest positive composite number is 4). Therefore, $(n - 1)!$ being the product of the numbers from 1 to $n - 1$ includes among its divisors the greatest prime factor of $n$, and indeed all its proper divisors. In fact, for composite $n > 4$, it is the case that $(n - 1)!$ not only has all the same proper divisors of $n$ as a subset of its own proper divisors, but has them with greater multiplicity than $n$ does. For the special case of $n = 4$, the congruence $(n - 1)! \equiv 2 \mod n$ is satisfied. For all larger composite $n$, the congruence $(n - 1)! \equiv 0 \mod n$ is satisfied instead of the congruence stated in the theorem.
\end{proof}

The special case of $n = 4$ deserves further special attention, as it is an exception which proves the rule. With any other semiprime $n = pq$, with either $p$ or $q$ being a prime greater than 2, the product $(n - 1)!$ contains, in addition to $p$ and $q$, both $(p - 1)q$ and $p(q - 1)$ (which are distinct numbers if $p \neq q$). So if $p$ and $q$ are distinct, then $(n - 1)!$ has both prime factors $p$ and $q$ with a multiplicity of at least 2, which is greater than the multiplicity of 1 in the semiprime $pq$. But with 4, the numbers $(p - 1)q$ and $p(q - 1)$ are both 2, and so 3! includes 2 as a factor with only a multiplicity of 1, which is less than that factor's multiplicity in 4.
\end{demark}

\begin{thebibliography}{1}
\bibitem{tk} Thomas Kochy, ''Elementary Number Theory with Applications'', 2nd Edition. London: Elsevier (2007): 324 - 325
\end{thebibliography}
%%%%%
%%%%%




\end{cnl}
%%%%%
%%%%%
\end{document}
