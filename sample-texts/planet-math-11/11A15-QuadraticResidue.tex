\documentclass[12pt]{article}
\usepackage{pmmeta}
\pmcanonicalname{QuadraticResidue}
\pmcreated{2013-03-22 11:55:19}
\pmmodified{2020-02-12}
\pmowner{mathcam}{2727}
\pmmodifier{mathcam}{2727}
\pmformalizer{ngonga}{}
\pmtitle{quadratic residue}
\pmrecord{9}{30622}
\pmprivacy{1}
\pmauthor{mathcam}{2727}
\pmtype{Definition}
\pmcomment{trigger rebuild}
\pmclassification{msc}{11A15}
\pmrelated{LegendreSymbol}
\pmrelated{EulersCriterion}
\pmdefines{quadratic non-residue}
\pmdefines{quadratic nonresidue}

\endmetadata

\usepackage{amssymb}
\usepackage{amsmath}
\usepackage{amsfonts}
\usepackage{graphicx}
%%%%\usepackage{xypic}

\usepackage{cnl}
\usepackage{xcolor}


% TITLE AUTHOR DATE
\title{Number Theory,\\ QuadraticResidue}
\date{February 12, 2020}
\author{Ngo Nga (ngongatlu)}



%- DOCUMENT

\begin{document}
\parskip=\baselineskip

% Local Defs 
\def\natdiv#1#2{{#1}\mathrel{|}{#2}}
\parindent=0pt

\begin{cnl}

\Cnlinput{../TeX2CNL/package/cnlinit}

\bigskip

%-% BEGIN


\lsection{QuadraticResidue}
In this section, let $d,\ m,\ p,\ q,\ a$ be integers.

%-% Definition

\dfn {We say $d$ \df{divides} $m$ iff $d\ne 0$ and there exists an integer $r$
  such that $m= d\*r$.}
			
We write  $\natdiv{d}{m}$ iff $d$ divides $m$.		
		
We say  $d$ is a \df{divisor of} $m$ iff $d$ divides $m$.

\dfn{We say $d$ is a \df{common\~divisor} of $p$ and $q$ iff $d$
  divides $p$ and $d$ divides $q$.}


\begin{remark}
We now introduce the greatest common divisor: 
\end{remark}

\dfn{We say $d$ is a \df{greatest\~common\~divisor} of $p$ and $q$  
iff $d$ is a common\~divisor of $p$ and $q$ such that 
for every common\~divisor $r$ of $p$ and $q$ we have $r\leq d$.}

\dfn{ Let $\h{\df{gcd}}\ p\ q$ denote the greatest\~common\~divisor of
  $p$ and $q$. This exists and is unique.}

\dfn
{We say that $p,\ q$ are  \df{coprime} iff  $gcd\ p\ q =1$. }

\dfn
{We say that $p,\ q$ are \df{relatively\~prime} iff $p,\ q$ are coprime.}


\dfn{
Assume that $a,\ n$ are integers. Assume that $a,\ n$ are relatively\~prime. 
We say that $a$ is a \df{quadratic\~residue} of $n$ iff there exists an integer
 $x$ such that $n$ divides $x\* x - a$, else, we say that $a$ is a \df{quadratic\~nonresidue} of $n$.
}
\end{cnl}





%%%%%
%%%%%
%%%%%
\end{document}
