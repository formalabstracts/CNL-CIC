\documentclass[12pt]{article}
\usepackage{pmmeta}
\pmcanonicalname{DivisibilityByPrimeNumber}
\pmcreated{2013-03-22 14:48:18}
\pmmodified{2020-02-12}
\pmowner{pahio}{2872}
\pmmodifier{pahio}{2872}
\pmformalizer{ngonga}{}
\pmtitle{divisibility by prime number}
\pmrecord{18}{36460}
\pmprivacy{1}
\pmauthor{pahio}{2872}
\pmtype{Theorem}
\pmcomment{trigger rebuild}
\pmclassification{msc}{11A05}
\pmsynonym{divisibility by prime}{DivisibilityByPrimeNumber}
\pmrelated{PrimeElement}
\pmrelated{DivisibilityInRings}
\pmrelated{EulerPhiAtAProduct}
\pmrelated{RepresentantsOfQuadraticResidues}

\pmdefines{divides}
\pmdefines{divisor}
\pmdefines{prime}

\endmetadata


\usepackage{amssymb}
\usepackage{amsmath}
\usepackage{amsfonts}

\usepackage{cnl}
\usepackage{xcolor}

% TITLE AUTHOR DATE
\title{Number Theory,\\ {DivisibilityByPrimeNumber}
\date{February 12, 2020}
\author{Ngo Nga (ngongatlu)}

%- DOCUMENT

\begin{document}
\parskip=\baselineskip

% Local Defs 
\def\natdiv#1#2{{#1}\mathrel{|}{#2}}
\parindent=0pt

\begin{cnl}

\Cnlinput{../TeX2CNL/package/cnlinit}

\bigskip

%-% BEGIN

\lsection{DivisibilityByPrimeNumber}
In this section, let $m,\ d$ be integers.


\deflabel{divides}
Assume that $m, \d$ are integers. Then we say that $d$ \df{divides} $m$ iff $d\ne 0$ and there exists an integer $r$ such that $m= d\*r$.
\end{definition}

\dfn{
We write $\natdiv{d}{m}$ iff $d$ divides $m$.
}

\dfn{
We say that $d$ is a \df{divisor of} $m$ iff $d$ divides $m$.
}

\dfn{
We say that $d$ is \df{positive} iff $d > 0$.
}

\deflabel{prime}
Assume that $p$ is a natural number greater than $1$.  Then we say that $p$ is a \df{prime}
 iff each positive divisor of $p$ is equal to $1$ or equal to $p$.
\end{definition}

\begin{theorem}
 Assume that $a$ and $b$ are integers and $p$ is a prime number. If $p \mid ab$ then $p \mid a$ or
 $p \mid b$.
\end{theorem}

\begin{demark}
{\em Proof.}  Suppose that\, $p \mid ab$.\,  Then either\; $p \mid a$\; or\; $p \nmid a$.\, In the
 latter case we have\, $\gcd(a,\,p) = 1$,\, and therefore the corollary of B\'ezout's lemma gives
 the result\, $p \mid b$.\,  Conversely, if\; $p \mid a$\; or\; $p \mid b$,\, then for example\,
 $a = mp$\, for some integer $m$; this implies that\, $ab = mb\cdot p$,\, i.e.\, $p \mid ab$.\\

\textbf{Remark 1.}  The theorem means, that if a product is divisible by a prime number, then at 
least one of the factor is divisibe by the prime number.  Also conversely.\\

\textbf{Remark 2.}  The condition (1) is expressed in \PMlinkescapetext{terms} of principal ideals as
\begin{align}
      (ab)\subseteq (p) \quad \Leftrightarrow \quad 
          (a)\subseteq (p)\, \lor\, (b)\subseteq (p).
\end{align}
Here, $(p)$ is a prime ideal of $\mathbb{Z}$.
\end{demark}


\end{cnl}
%%%%%
%%%%%
\end{document}
