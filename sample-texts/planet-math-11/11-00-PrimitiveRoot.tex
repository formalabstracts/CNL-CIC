\documentclass[12pt]{article}
\usepackage{pmmeta}
\pmcanonicalname{PrimitiveRoot}
\pmcreated{2013-03-22 16:04:33}
\pmmodified{2013-03-22 16:04:33}
\pmowner{CWoo}{3771}
\pmmodifier{CWoo}{3771}
\pmformalizer{NguyenNhung}{hoangnhung227}
\pmtitle{primitive root}
\pmrecord{12}{38133}
\pmprivacy{1}
\pmauthor{CWoo}{3771}
\pmtype{Definition}
\pmcomment{trigger rebuild}
\pmclassification{msc}{11-00}
\pmsynonym{primitive root modulo n}{PrimitiveRoot}
\pmsynonym{primitive element}{PrimitiveRoot}
\pmrelated{MultiplicativeOrderOfAnIntegerModuloM}
\pmrelated{PrimeResidueClass}
\pmrelated{UsingPrimitiveRootsAndIndexToSolveCongruences}

\endmetadata

% this is the default PlanetMath preamble.  as your knowledge
% of TeX increases, you will probably want to edit this, but
% it should be fine as is for beginners.

% almost certainly you want these
\usepackage{amssymb}
\usepackage{amsmath}
\usepackage{amsfonts}

\usepackage{cnl}
\usepackage{xcolor}


% TITLE AUTHOR DATE
\title{Number Theory,\\ 11-00-PrimitiveRoot}
\date{February 27, 2020}
\author{Nguyen Nhung (hoangnhung227)}

% there are many more packages, add them here as you need them

% define commands here
\newcommand{\intpow}[2]{{#1}^{#2}}

\begin{document}

\parskip=\baselineskip

\begin{cnl}
\Cnlinput{../TeX2CNL/package/cnlinit}

\bigskip

In this section, let $r$, $n$ be integers.

\begin{remark}
	\item We need divisibility concept.
	\item We need greatest common divisor (GCD) concept.
	\item We need prime concept
	\item We need relatively prime
\end{remark}

%Define Congruent Modulo
\dfn{
	Assume that $b, c$ are integers. Suppose $m$ is a positive integer. Then we say that $b$ is congruent to $c$ modulo $m$ iff m divides their difference b-c.
}

We write $ b \equiv c \mod m$ iff $b$ is congruent to $c$ modulo $m$.

We write $ b \notequiv c \mod m$ iff $b$ is not congruent to $c$ modulo $m$.

 %Define Euler's totient function
\dfn{ Assume that $n$ is a positive integer. Then Euler's totient function of n is a number of the positive integers up to n that are relatively prime to n.
}
Let $\phi(n)$ denote Euler's totient function of $n$.

%define reduced residue system modulo
\dfn{
	Assume that $n$ is a positive integer. Let $\phi(n)$ be Euler's function. A reduced residue system modulo n is a set of $\phi(n)$ integers with
	\item for all $x$, $gcd(x,m) = 1$
	\item for all $x \y$,  $x \notequiv y \mod m$.
}

% define primitive root modulo
\dfn{We say that $r$ is a \df{primitive root modulo} $n$ iff the numbers $r^0,\,r^1,\,\ldots,\,r^{n-2}$ form a reduced residue system modulo $n$.}

\begin{remark}
For example, 2 is a primitive root modulo 5, since
$1,\; 2,\; 2^2 = 4,\; 2^3 = 8 \equiv 3 \pmod{5}$
are all with 5 coprime positive integers less than 5.\\
\end{remark}

\begin{remark}
The generalized Riemann hypothesis implies that every prime number $p$ has a primitive root below $70(\ln p)^2$.
\end{remark}

%%%%%
%%%%%
\end{cnl}
\end{document}
