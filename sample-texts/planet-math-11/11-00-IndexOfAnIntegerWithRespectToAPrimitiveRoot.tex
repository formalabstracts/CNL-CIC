\documentclass[12pt]{article}
\usepackage{pmmeta}
\pmcanonicalname{IndexOfAnIntegerWithRespectToAPrimitiveRoot}
\pmcreated{2013-03-22 16:20:50}
\pmmodified{2013-03-22 16:20:50}
\pmowner{alozano}{2414}
\pmmodifier{alozano}{2414}
\pmformalizer{NguyenNhung}{hoangnhung227}
\pmtitle{index of an integer with respect to a primitive root}
\pmrecord{4}{38480}
\pmprivacy{1}
\pmauthor{alozano}{2414}
\pmtype{Definition}
\pmcomment{trigger rebuild}
\pmclassification{msc}{11-00}
\pmrelated{PrimitiveRoot}
\pmdefines{primitive root}
\pmdefines{relatively prime}
\\pmdefines{modulo}
\endmetadata

% this is the default PlanetMath preamble.  as your knowledge
% of TeX increases, you will probably want to edit this, but
% it should be fine as is for beginners.

% almost certainly you want these
\usepackage{amssymb}
\usepackage{amsmath}
\usepackage{amsfonts}

\usepackage{cnl}
\usepackage{xcolor}


% TITLE AUTHOR DATE
\title{Number Theory,\\ 11-00-IndexOfAnIntegerWRTPrimitiveRoot}
\date{February 27, 2020}
\author{Nguyen Nhung (hoangnhung227)}

% there are many more packages, add them here as you need them

% define commands here
\newcommand{\intpow}[2]{{#1}^{#2}}

\begin{document}

\parskip=\baselineskip

\begin{cnl}
\Cnlinput{../TeX2CNL/package/cnlinit}

\bigskip

In this section, let $a$, $b$, $c$, $m$, $n$, $g$, $d$ be integers.

 \begin{remark}
 	\item We need divisibility concept
 	\item We need greatest common divisor (GCD) concept
 	\item We need prime concept
 	\item We need relatively prime
 \end{remark}
 
 %Define Congruent Modulo
 \dfn{
 	Assume that $b, c$ are integers. Suppose $m$ is a positive integer. Then we say that $b$ is congruent to $c$ modulo $m$ iff m divides their difference b-c.
 }
 
 We write $ b \equiv c \mod m$ iff $b$ is congruent to $c$ modulo $m$.
 
 We write $ b \notequiv c \mod m$ iff $b$ is not congruent to $c$ modulo $m$.
 
 %Define Euler's totient function
 \dfn{ Assume that $n$ is a positive integer. Then Euler's totient function of n is a number of the positive integers up to n that are relatively prime to n.
 }
 Let $\phi(n)$ denote Euler's totient function of $n$.
 
 %define reduced residue system modulo
 \dfn{
 	Assume that $n$ is a positive integer. Let $\phi(n)$ be Euler's totient function. A reduced residue system modulo n is a set of $\phi(n)$ integers with
 	\item for all $x$, $gcd(x,m) = 1$
 	\item for all $x \y$,  $x \notequiv y \mod m$.
 }
 
 % define primitive root modulo
 \dfn{We say that $r$ is a \df{primitive root modulo} $n$ iff the numbers $r^0,\,r^1,\,\ldots,\,r^{n-2}$ form a reduced residue system modulo $n$.}
 
% define index of an integer wrt primitive root
\dfn{Assume that $m>1$ is an integer such that the integer $g$ is a primitive root for $m$. Suppose $a$ is another integer relatively prime to $g$. Then we say that $n$ is index of $a$ to base $g$ iff $n$ is the smallest positive integer such that $g^n\equiv a \mod m$, and it is denoted by $\operatorname{ind} a$ or $\operatorname{ind}_g a$.
}

\begin{remark}
If $m$ has a primitive root the index with respect to a primitive root is a very useful tool to solve polynomial congruences modulo $m$.
\end{remark}

\end{cnl}

%%%%%
%%%%%
\end{document}
	