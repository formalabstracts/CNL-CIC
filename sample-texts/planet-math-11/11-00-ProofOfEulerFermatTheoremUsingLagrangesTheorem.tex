\documentclass[12pt]{article}
\usepackage{pmmeta}
\pmcanonicalname{ProofOfEulerFermatTheoremUsingLagrangesTheorem}
\pmcreated{2013-03-22 14:24:03}
\pmmodified{2013-03-22 14:24:03}
\pmowner{alozano}{2414}
\pmmodifier{alozano}{2414}
\pmformalizer{NguyenNhung}{hoangnhung227}
\pmtitle{proof of Euler-Fermat theorem using Lagrange's theorem}
\pmrecord{4}{35898}
\pmprivacy{1}
\pmauthor{alozano}{2414}
\pmtype{Proof}
\pmcomment{trigger rebuild}
\pmclassification{msc}{11-00}
\pmrelated{LagrangesTheorem}
\pmrelated{FermatsLittleTheorem}
\pmrelated{FermatsTheoremProof}

\endmetadata

% this is the default PlanetMath preamble.  as your knowledge
% of TeX increases, you will probably want to edit this, but
% it should be fine as is for beginners.

% almost certainly you want these
\usepackage{amssymb}
\usepackage{amsmath}
\usepackage{amsthm}
\usepackage{amsfonts}

\usepackage{cnl}
\usepackage{xcolor}


% TITLE AUTHOR DATE
\title{Number Theory,\\ 11-00-ProofOfEulerFermatTheoremUsingLagrangesTheorem}
\date{March 18, 2020}
\author{Nguyen Nhung (hoangnhung227)}

% there are many more packages, add them here as you need them

% define commands here
\newcommand{\intpow}[2]{{#1}^{#2}}

% LOCAL DEFS FOR SYLOWS THEOREMS

% used for TeXing text within eps files
%\usepackage{psfrag}
% need this for including graphics (\includegraphics)
%\usepackage{graphicx}
% for neatly defining theorems and propositions
%\usepackage{amsthm}
% making logically defined graphics
%%%\usepackage{xypic}

% there are many more packages, add them here as you need them

% define commands here

%- THEOREMS
\newtheorem{definition}{Definition}
\newtheorem{theorem}[definition]{Theorem}
\newtheorem{lemma}[definition]{Lemma}
\numberwithin{definition}{section}
%\newtheorem{specification}[definition]{Specification}

% Some sets
\newcommand{\Nats}{\mathbb{N}}
\newcommand{\Ints}{\mathbb{Z}}
\newcommand{\Reals}{\mathbb{R}}
\newcommand{\Complex}{\mathbb{C}}
\newcommand{\Rats}{\mathbb{Q}}

\begin{document}
	
	\parskip=\baselineskip
	
	\begin{cnl}
		\Cnlinput{../TeX2CNL/package/cnlinit}
		
		\bigskip
		
		In this section, let $a$, $n$ be integers.

\begin{remark}
	\begin{itemize}
		\item We need divisibility concept.
		\item We need greatest common divisor (GCD) concept.
		\item We need congruent modulo concept.
		\item We need the Euler totient function.
	\end{itemize}
\end{remark}


\begin{theorem}
	Assume that $a, n$ are positive integers and $\phi(n)$ is Euler's function. If $gcd \ a\ n =1$  then $a^{\phi (n)} \equiv 1 \mod n$.
\end{theorem}

\begin{remark}
\begin{proof}
We will make use of Lagrange's Theorem: Let $G$ be a finite group and let $H$ be a subgroup of $G$. Then the order of $H$ divides the order of $G$.

Let $G=(\Ints/n\Ints)^\times$ and let $H$ be the multiplicative subgroup of $G$ generated by $a$ (so $H=\{ 1, a ,a^2,\ldots \}$). The fact that $\gcd(a,n)=1$ ensures that $a\in G$. Notice that the order of $H$, $h=|H|$ is also the order of $a$, i.e. the smallest natural number $n>1$ such that $a^n$ is the identity in $G$, i.e. $a^h\equiv 1 \mod n$. Also, recall that the order of $G=(\Ints/n\Ints)^\times$ is $\phi(n)$, where $\phi$ is the Euler $\phi$ function.

By Lagrange's theorem $h \mid |G|=\phi(n)$, so $\phi(n)=h\cdot m$ for some $m$. Thus:
$$a^{\phi(n)}=(a^h)^m\equiv 1^m \equiv 1 \mod n$$
as claimed.

\end{proof}
\end{remark}
%%%%%
%%%%%
\end{cnl}
\end{document}
