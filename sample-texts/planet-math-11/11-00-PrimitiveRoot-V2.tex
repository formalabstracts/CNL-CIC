\documentclass[12pt]{article}
 \usepackage{pmmeta}
 \pmcanonicalname{PrimitiveRoot}
 \pmcreated{2013-03-22 16:04:33}
 \pmmodified{2013-03-22 16:04:33}
 \pmowner{CWoo}{3771}
 \pmmodifier{CWoo}{3771}
 \pmformalizer{NguyenNhung}{hoangnhung227}
 \pmtitle{primitive root}
 \pmrecord{12}{38133}
 \pmprivacy{1}
 \pmauthor{CWoo}{3771}
 \pmtype{Definition}
 \pmcomment{trigger rebuild}
 \pmclassification{msc}{11-00}
 \pmsynonym{primitive root modulo n}{PrimitiveRoot}
 \pmsynonym{primitive element}{PrimitiveRoot}
 \pmrelated{MultiplicativeOrderOfAnIntegerModuloM}
 \pmrelated{PrimeResidueClass}
 \pmrelated{UsingPrimitiveRootsAndIndexToSolveCongruences}

  \endmetadata

  % this is the default PlanetMath preamble.  as your knowledge
 % of TeX increases, you will probably want to edit this, but
 % it should be fine as is for beginners.

  % almost certainly you want these
 \newcommand{\notequiv}{\not\equiv}
 \usepackage{amssymb}
 \usepackage{amsmath}
 \usepackage{amsfonts}

  \usepackage{cnl}
 \usepackage{xcolor}

 
  % TITLE AUTHOR DATE
 \title{Number Theory,\\ 11-00-PrimitiveRoot}
 \date{February 27, 2020}
 \author{Nguyen Nhung (hoangnhung227)}

  % there are many more packages, add them here as you need them

  % define commands here
 \newcommand{\intpow}[2]{{#1}^{#2}}

  % LOCAL DEFS FOR SYLOWS THEOREMS

 
  \begin{document}

  \parskip=\baselineskip

  \begin{cnl}
 \Cnlinput{../TeX2CNL/package/cnlinit}

  \bigskip

  In this section, let $a$, $b$, $c$, $d$, $m$, $n$, $r$ be integers.

  \begin{remark}
 	\begin{itemize}
 	\item We need divisibility concept.
 	\item We need greatest common divisor (GCD) concept.
 	\item We need prime concept.
 	\item We need relatively prime.
 	\end{itemize}
 \end{remark}

  %Define Congruent Modulo
 \dfn{
 	Assume that $b, c$ are integers. Suppose $m$ is a positive integer. Then we say that $b$ is congruent to $c$ modulo $m$ iff m divides their difference b-c.
 }

  We write $ b \equiv c \mod m$ iff $b$ is congruent to $c$ modulo $m$.

  We write $ b \notequiv c \mod m$ iff $b$ is not congruent to $c$ modulo $m$.

  % define primitive root modulo
 \dfn{We say that $r$ is a \df{primitive root modulo} $m$ iff for every integer $n$ that is relatively prime to $m$, there exists a natural number $i$ such that $ r^i \equiv n \mod m$.
 }

  \begin{remark}
 For example, 2 is a primitive root modulo 5, since
 $1,\; 2,\; 2^2 = 4,\; 2^3 = 8 \equiv 3 \pmod{5}$
 are all with 5 coprime positive integers less than 5.\\
 \end{remark}

  \begin{remark}
 The generalized Riemann hypothesis implies that every prime number $p$ has a primitive root below $70(\ln p)^2$.
 \end{remark}

  %%%%%
 %%%%%
 \end{cnl}
 \end{document}