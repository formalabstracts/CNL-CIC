\documentclass[12pt]{article}
\usepackage{pmmeta}
\pmcanonicalname{SierpinskiConjecture}
\pmcreated{2013-03-22 13:34:16}
\pmmodified{2019-12-06}
\pmowner{yark}{2760}
\pmmodifier{yark}{2760}
\pmformalizer{chaunguyen}{}
\pmtitle{Sierpi\'nski conjecture}
\pmrecord{12}{34184}
\pmprivacy{1}
\pmauthor{yark}{2760}
\pmtype{Conjecture}
\pmcomment{trigger rebuild}
\pmclassification{msc}{11B83}
\pmsynonym{Sierpinski conjecture}{SierpinskiConjecture}
\pmdefines{composite}

\endmetadata


\usepackage{amssymb}
\usepackage{amsmath}
\usepackage{amsfonts}
\usepackage{cnl}
\usepackage{xcolor}
%\usepackage{graphicx}
%%%%\usepackage{xypic}

%- DOCUMENT


\begin{document}
\parskip =\baselineskip
	
% Local Defs 
\def\natdiv#1#2{{#1}\mathrel{|}{#2}}
\def\intpow#1#2{{#1}^{#2}}


\begin{cnl}
		
\Cnlinput{../Tex2CNL/package/cnlinit}
		
\bigskip
		
%-% BEGIN

\lsection{SierpinskiConjecture}

In this section, let $d,\ m,\ p,\ q$ be integers.

%-% Definition

\dfn{ We say that $d$ is \df{positive } iff $d > 0$. }
		
\dfn{ We say that $d$ \df{divides} $m$ iff $d\ne 0$ and there
 exists an integer $r$ such that $m= d\*r$. }
	
\dfn{ We write  $\natdiv{d}{m}$ iff $d$ divides $m$. }

\dfn{ We say  $d$ is a \df{divisor} of $m$ iff $d$ divides $m$. }

\dfn{ Assume that $p$ is greater than $1$.  Then we say that $p$
 is a \df{prime} iff each positive divisor of $p$ is equal to $1$
 or equal to $p$. }

\dfn{ We say $k$ is a composite iff $k$ is a positive integer and
 $k$ is not a prime. }

\dfn{ We say that $d$ is an \df{odd number} iff there exists an
 integer $m$ such that $d=2\*m+1$. }

\dfn{ Assume that $b$ is an integer and $n$ is a natural number.  
Then we define
$\intpow{b}{n}\assign$
\par$\match\ n$ with
\begin{envMatch}
	\firstmatchitem $0$ &$\assign$& $1$
	\matchitem $m + 1$ &$\assign$& $b \* \intpow{b}{m}$\texstop
\end{envMatch}\cnlstop
This exists by recursion.
}

\dfn{ We say that $k$ is a \df{Sierpinski number} iff $k$ is an 
odd integer and for every integer $n$ greater than $1$ we have 
$k\*\intpow {2}{n} + 1$ is a composite. }

\begin{remark}
In 1960 Wac{\l}aw Sierpi\'nski (1882-1969) proved the following 
interesting result:
\end{remark}

\begin{theorem}
For every Sierpinski number $k$ there exists some Sierpinski number 
$r$ such that $r>k$. 
\end{theorem}

\begin{remark}
The Sierpi\'nski problem consists in determining the smallest 
Sierpi\'nski number. In 1962, John Selfridge discovered the 
Sierpi\'nski number $k = 78557$, which is now believed to be in fact
 the smallest such number. 
\end{remark}

\begin{conjecture}
$78557$ is a Sierpi\'nski number and there exists no Sierpi\'nski 
number smaller than $78557$.
\end{conjecture}

\begin{remark}
To prove the conjecture, it would be sufficient to exhibit
a prime $k2^n+1$ for each $k < 78557$.
\end{remark}


\end{cnl}


\end{document}