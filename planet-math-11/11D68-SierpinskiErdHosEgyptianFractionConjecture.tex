\documentclass[12pt]{article}
\usepackage{pmmeta}
\pmcanonicalname{SierpinskiErdHosEgyptianFractionConjecture}
\pmcreated{2013-03-22 13:43:12}
\pmmodified{2019-12-06}
\pmowner{CWoo}{3771}
\pmmodifier{CWoo}{3771}
\pmformalizer{chaunguyen}{}
\pmtitle{Sierpinski Erd\H{o}s egyptian fraction conjecture}
\pmrecord{8}{34403}
\pmprivacy{1}
\pmauthor{CWoo}{3771}
\pmtype{Conjecture}
\pmcomment{trigger rebuild}
\pmclassification{msc}{11D68}
\pmclassification{msc}{11A67}
\pmsynonym{Sierpi\'nski Erd\H{o}s egyptian fraction conjecture}{SierpinskiErdHosEgyptianFractionConjecture}
%\pmkeywords{egyptian fraction}
\pmrelated{UnitFraction}
\pmrelated{AnyRationalNumberWithOddDenominatorIsASumOfUnitFractionsWithOddDenominators}

\endmetadata



\usepackage{amssymb}
\usepackage{amsmath}
\usepackage{amsfonts}
\usepackage{cnl}
\usepackage{xcolor}
%\usepackage{graphicx}
%%%%\usepackage{xypic}

%- DOCUMENT


\begin{document}
\parskip =\baselineskip
	
% Local Defs 

\begin{cnl}
		
\Cnlinput{../Tex2CNL/package/cnlinit}
		
\bigskip
		
%-% BEGIN

\lsection{SierpinskiErdHosEgyptianFractionConjecture}

In this section, let $d$ be an integer.

%-% Definition

\dfn{ We say that $d$ is \df{positive } iff $d > 0$. }

\begin{remark}
Erd\H{o}s and Sierpinski conjectured that:  
\end{remark}

\begin{conjecture}
For any integer $ n$ greater than 3 there exist 
positive integers $a,\ b,\ c $ such that $ 5/n=1/a + 1/b + 1/c $. 
\end{conjecture}


\end{cnl}


\end{document}