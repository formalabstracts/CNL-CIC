
\documentclass[12pt]{article}


%- PACKAGES
\usepackage{url}
\usepackage{amsmath}
\usepackage{amsthm}
\usepackage{amssymb}
\usepackage{xcolor}
\usepackage{currfile}
\usepackage{fancyvrb}
\usepackage{xparse} % needed for \ellipsis control sequence in cnl-style
\usepackage{enumitem} % for topsep=0pt
\usepackage{colonequals}

% for underscores https://texfaq.org/FAQ-underscore
\usepackage{lmodern}
\usepackage[T1]{fontenc}
\usepackage{textcomp}
\usepackage{lineno}

\usepackage[
bookmarksopen,
bookmarksdepth=2,
%breaklinks=true
colorlinks=true,
urlcolor=blue]{hyperref}

% GLOBAL FORMATTING
%\linenumbers
\parindent=0pt
\parskip=0.5\baselineskip
\raggedbottom

% TITLE AUTHOR DATE
\title{Number Theory,\\ Even and odd number}

\date{October 18, 2019}
\author{My Nguyen (miinguyen)}

%- THEOREMS
\newtheorem{definition}{Definition}
\newtheorem{theorem}[definition]{Theorem}
\newtheorem{lemma}[definition]{Lemma}
\numberwithin{definition}{section}
%\newtheorem{specification}[definition]{Specification}


%- DOCUMENT

\begin{document}
\maketitle

\setcounter{tocdepth}{1}
\tableofcontents
\newpage


%\renewcommand\labelitemi{\normalfont|}}


%% ELLIPSIS, 
% based on Luis Berlioz's query
% https://tex.stackexchange.com/questions/503731/how-to-define-a-macro-that-takes-the-definition-of-a-macro-as-an-argument
% \usepackage{amsmath}
% \usepackage{xparse}

\ExplSyntaxOn
\NewDocumentCommand{\ellipsis}{mmmm}
 {% #1 = main term
  % #2 = first index
  % #3 = last index
  % #4 = operation
  \group_begin:
  \lucas_ellipsis:nnnn { #1 } { #2 } { #3 } { #4 }
  \group_end:
 }
\cs_new:Nn \lucas_ellipsis:nnnn
 {
  \cs_set:Nn \__lucas_ellipsis_term:n { #1 }
  \__lucas_ellipsis_term:n { #2 }
  #4 \dots #4
  \__lucas_ellipsis_term:n { #3 }
 }
\ExplSyntaxOff

%\begin{document}

%$\ellipsis{x^{#1}}{0}{5}{+}$

%$\ellipsis{x_{#1}}{0}{5}{+}$

%$\ellipsis{(x_{#1}+y_{#1}i)}{1}{n}{}$

%\end{document}
%

%% SUSPEND itemize
% https://tex.stackexchange.com/questions/135726/intertext-like-command-in-enumerate-environment


%% CNL CONTROL SEQUENCES

\def\ignoreOptionAndCS[#1]#2{}

\def\onearg#1{(onearg:#1)} % for debugging.

\ignoreOptionAndCS[3]\onearg


\def\ignoreOne#1{}

%\ignoreOne\onearg C

\def\ignoreOptionOrCS{%
\futurelet\nextToken\chooseBranch}


\def\chooseBranch{%
\let\next=\relax
\ifx\nextToken [%
 \let\next=\ignoreOptionAndCS%
\else%
 \let\next=\ignoreOne%
\fi%
\next%
}
\ignoreOne{]}

\def\ignoreOptionAndCSS[#1]#2#3{}
\def\ignoreTwo#1#2{}
\def\ignoreOptionOrCSS{%
\futurelet\nextToken\chooseBranchTwo%
}
\def\chooseBranchTwo{%
\let\next=\relax
\ifx\nextToken [%
 \let\next=\ignoreOptionAndCSS%
\else%
 \let\next=\ignoreTwo%
\fi%
\next%
}

\let\CnlExpand=\ignoreOptionOrCS
\let\CnlNoExpand=\ignoreOptionOrCS
\let\CnlDelete=\ignoreOptionOrCS
\let\CnlCustom=\ignoreOptionOrCSS
\let\CnlDef=\ignoreOptionOrCSS
\let\CnlError=\ignoreOptionOrCS
\def\CnlEnvirDelete#1{}


%% ENVIRONMENTS

\newenvironment{cnl}{\centerline{\bf CNL text starts here.}}
{\centerline{\bf CNL text ends here.}}
% \leavevmode\par is to make remark work 
% when it is the first item in a subsection.

\newenvironment{remark}%{}{} %temp debug
{\leavevmode\par\begin{tabular}{|p{13cm}}\parskip=\baselineskip{\bf Remark.}}
{\end{tabular}}

\newenvironment{envMatch}%
               {\par\begin{tabular}{@{\quad\normalfont| }lll}}%
               {\end{tabular}\par}

\newenvironment{structure}%
{\begin{itemize}[topsep=0pt]}
{\end{itemize}}

\newcommand{\interitem}[1]{\end{itemize}#1\begin{itemize}[topsep=0pt]}


\newenvironment{make}%                             
{\begin{itemize}[topsep=0pt]}
{\end{itemize}}

%\newenvironment{format}%
%{\begin{align*}}{\end{align*}}

% ENVIRONMENT ITEMS
\renewcommand\labelitemi{--}
\newcommand{\matchitem}{\\}
\newcommand{\firstmatchitem}{}
\newcommand{\firstitem}{\item}

% LABELS
\def\lsection#1{\section{#1}\label{#1}}
\def\lsubsection#1{\subsection{#1}\label{#1}}
\def\lsubsubsection#1{\subsubsection{#1}\label{#1}}
\def\deflabel#1{\begin{definition}[#1]\label{#1}}
\def\thmlabel#1{\begin{theorem}[#1]\label{#1}}
\def\namelabel#1{[#1]\label{#1}}

% SPECIALS
\newcommand{\var}[1]{#1}
\newcommand{\id}[1]{#1}
\newcommand{\prefix}[1]{}
\newcommand{\app}[1]{#1}
\newcommand{\CnlList}[1]{#1} % was list
\newcommand{\parenI}[1]{#1}
\newcommand{\df}[1]{\text{\bf #1}}
\newcommand{\h}[1]{\text{\color{red} #1}}
\newcommand{\Mid}{\mathrel{\|}}
\newcommand{\ignore}[1]{}
\newcommand{\parenthetical}[1]{(#1)}
\newcommand{\funmapsto}[2]{#1\mapsto #2}
\newcommand{\funalign}{}
\newcommand{\setcomp}[2]{\{#1 \mid #2\}}
\newcommand{\setenum}[1]{\{#1\}}
\newcommand{\wherearg}[1]{\where\ $\{#1\}$}
\newcommand{\optarg}[1]{\{#1\}}
\newcommand{\texstop}{.}
\newcommand{\texcomma}{,}
\newcommand{\cnlstop}{}
\newcommand{\plural}{/-}
\newcommand\caseif[2]{#1 & \ifcond\ #2}
\newcommand\caseotherwise[1]{#1&\otherwise}
\newcommand\onlyTeX[1]{#1}

% SYMBOLS
\newcommand{\join}{\lor}
\newcommand{\blank}{\h{\_}}
\renewcommand{\qed}{\ensuremath{\square}}
\newcommand{\inv}[1]{{#1}^{-1}}
\newcommand{\nullbrack}{[]}
\renewcommand{\*}{\,}
\renewcommand{\iff}{\leftrightarrow}
\newcommand{\Prop}{\text{\tt Prop}}
\newcommand{\Type}{\text{\tt Type}}
\newcommand{\Bool}{\text{\tt Bool}}
\newcommand{\assign}{\colonequals} % {colonequals}
\newcommand{\cons}{\mathrel{\coloncolon}} % {colonequals}
\newcommand{\Nat}{\ensuremath{{\mathbb N}}}
\newcommand{\Real}{\ensuremath{{\mathbb R}}}

% KEYWORDS
\newcommand{\keyword}[1]{{\text{\bf{#1}}}}
\newcommand{\Make}{\keyword{make}}
\newcommand{\fun}{\keyword{fun}}
\newcommand{\match}{\keyword{match}}
\newcommand{\ifcond}{\keyword{if}}
\newcommand{\thencond}{\keyword{then}}
\newcommand{\elsecond}{\keyword{else}}
\newcommand{\otherwise}{\keyword{otherwise}}
\newcommand{\function}{\keyword{function}}
\newcommand{\quot}{\keyword{quot}}
\newcommand{\etc}{\keyword{etc}}
\newcommand{\where}{\keyword{where}}

% ACCENTS
% Use these versions to fuse with variable name in CNL image.
% \mathcheck{c} --> V__c_mathcheck
% \check{c} --> \check{c} (function application)
\newcommand{\mathhat}{\hat}
\newcommand{\mathwidehat}{\widehat}
\newcommand{\mathcheck}{\check}
\newcommand{\mathtilde}{\tilde}
\newcommand{\mathwidetilde}{\widetilde}
\newcommand{\mathacute}{\acute}
\newcommand{\mathgrave}{\grave}
\newcommand{\mathdot}{\dot}
\newcommand{\mathddot}{\ddot}
\newcommand{\mathbreve}{\breve}
\newcommand{\mathbar}{\bar}
\newcommand{\mathvec}{\vec}

%\renewcommand{\_}{\textunderscore}
%\newcommand{\comment}[1]{\%- \nobreak{#1}}
%\renewcommand{\~}{\ }
%\newcommand{\Bool}{\ensuremath{{\mathbb {B}}}}
%\newcommand{\cons}{\mathrel{{\bf :\hskip-1.1pt:}}}
%\newcommand{\fld}{.} %use \h{. }
%\newcommand{\assign}{\mathrel{:=}} %{\mathrel{:\equiv}}
%\newcommand{\caseif}{{\bf if}}
%\newcommand{\caseotherwise{{\bf otherwise}}

% Local Defs for Sylows Theorems
\def\natdiv#1#2{{#1}\mathrel{|}{#2}}
\def\natpow#1#2{{#1}^{#2}}
\def\conj#1#2{{#1}{#2}{#1}^{-1}}
\def\mult#1#2{m}
\def\Syl#1#2{\operatorname{Syl}_{#1}(#2)}
\def\SylN#1#2{n_{#1}(#2)}
\def\Nz#1#2{|N(#1,#2)|}

\begin{cnl}


%% We start with some that will eventually be moved to a standard library.

\CnlCustom[1]\parenI{ (#1) }
\CnlCustom\*{*}
\CnlCustom[1]\lsection{Section \concat{}{#1} .}
\CnlCustom[1]\lsubsection{Subsection \concat{}{#1} .}
\CnlCustom[1]\lsubsubsection{Subsubsection \concat{}{#1} .}
\CnlCustom[1]\deflabel{\begin{definition}\label{#1}}
\CnlCustom[1]\df{ #1 }
\CnlCustom[1]\h{ #1 }
\CnlCustom\where{where}
\CnlCustom\ifcond{if}
\CnlCustom\otherwise{true}
\CnlCustom\elsecond{else}
\CnlCustom\where{where}
\CnlCustom\assign{ \concat{}{:=} }
\CnlCustom\cons{\concat{}{::}} 
\CnlCustom\plural{\concat{}{/-}}
\CnlCustom\wherearg{where}
\CnlCustom\fun{fun}
\CnlCustom[2]\funmapsto{(fun #1 \assign #2)}
%\funalign{A}&{\to}{B}\\{a}{\mapsto}{b}
\CnlCustom[6]\funalign{(fun (#4 : #1) \assign (#6 : #3))}
\CnlCustom\Prop{Prop}
\CnlCustom\Type{Type}
\CnlCustom\Bool{Bool}
\CnlCustom\iff{iff}
\CnlCustom\optarg{}
\CnlCustom\match{match}


\CnlCustom\function{function}
\CnlCustom\Make{make}
\CnlCustom\etc{\_}
\CnlCustom\Nat{Nat}
\CnlCustom\Real{Real}
\CnlCustom\quot{quot}
\CnlCustom\alpha{V--alpha}
\CnlCustom\beta{V--beta}
\CnlCustom\prefix{\concat}
\CnlCustom\cnlstop{.}
%\CnlCustom\nullbrack{\concat{}{[]}}
\CnlCustom\matchitem{\alt}
\CnlCustom\firstmatchitem{\alt}
\CnlCustom[2]\caseif{#2 \assign #1}
\CnlCustom[1]\caseotherwise{true \assign #1}
\CnlCustom\rightarrow{\imply}
%\CnlNoExpand[1]\section


% Prohibited control sequences. 
% That is, they should remain outside the cnl environment.
% We should list many TeX primitive control sequences here.

% CNLERROR
\CnlError\if
\CnlError\else
\CnlError\fi
\CnlError\let
\CnlError\futurelet
\CnlError\afterassignment
\CnlError\usepackage
% etc. 

% CNLDELETE
\CnlDelete\expandafter
\CnlDelete[1]\interitem
\CnlDelete\texstop
\CnlDelete[1]\onlyTeX
\CnlDelete[1]\phantom
\CnlDelete\firstitem
\CnlDelete\texstop
\CnlDelete\texcomma
% math modes and space
\CnlDelete\ensuremath
\CnlDelete\text
\CnlDelete\thinmuskip
\CnlDelete\medmuskip
\CnlDelete\thickmuskip
\CnlDelete\quad
\CnlDelete\qquad
\CnlDelete\,
\CnlDelete\:
\CnlDelete\;
\CnlDelete\!
\CnlDelete\ %space
\CnlDelete\enspace
\CnlDelete[1]\hspace
\CnlDelete\hfil
\CnlDelete\hfill
\CnlDelete\thinspace
\CnlDelete\left
\CnlDelete\right
\CnlDelete\big
\CnlDelete\Big
\CnlDelete\bigg
\CnlDelete\Bigg
\CnlDelete\allowdisplaybreaks
%other  spacing
\CnlDelete\noindent
\CnlDelete\indent
\CnlDelete[1]\vspace
\CnlDelete\null
\CnlDelete\break
\CnlDelete\newline
\CnlDelete\newpage
\CnlDelete\vfil
\CnlDelete\vfill
\CnlDelete\smallskip
\CnlDelete\medskip
\CnlDelete\bigskip
\CnlDelete[2]\rule
\CnlDelete[1]\parenthetical

%\CnlDelete\par

\CnlEnvirDelete{remark}
\CnlEnvirDelete{summary}
\CnlEnvirDelete{tikzpicture}
\CnlEnvirDelete{fancyvrb}




%-% BEGIN
\lsection{Even Number}

In this section, let $k,\ r$ be integers.

%-% Definition
\deflabel{even number}
Assume that $k$ is an integer number.  We say that $k$
is a \df{even number} iff there exists $r$ such that $k=2\*r$
\end{definition}

\deflabel{odd number}
Assume that $k$ is an integer number.  We say that $k$
is a \df{odd number} iff there exists $r$ such that $k=2\*r+1$
\end{definition}

\begin{remark}
The numbers
The concept of \h{even number} and \h{odd numbers} are most easily understood in 
the binary base. Then the above definition simply that even numbers end
with a $0$, and odd numbers end with a $1$.
\end{remark}

\begin{theorem}[Prop1]
Every integer is either even or odd.
\end{theorem}

\begin{remark}
This theorem above can be proven using induction, or using the fundamental theorem of arithmetic.
\end{remark}

\begin{theorem}[Prop2]
An integer $k$ is even iff $\natpow{p}{2}$ is even.
\end{theorem}


\end{cnl}




\end{document}