\documentclass[12pt]{article}
\usepackage{pmmeta}
\pmcanonicalname{Interprime}
\pmcreated{2013-03-22 18:08:25}
\pmmodified{2019-12-06}
\pmowner{PrimeFan}{13766}
\pmmodifier{PrimeFan}{13766}
\pmformalizer{chaunguyen}{}
\pmtitle{interprime}
\pmrecord{5}{40695}
\pmprivacy{1}
\pmauthor{PrimeFan}{13766}
\pmtype{Definition}
\pmcomment{trigger rebuild}
\pmclassification{msc}{11A51}
\pmrelated{MinimalAndMaximalNumber}
\pmdefines{interprime}
\pmdefines{consecutive primes}
\pmdefines{twin primes}
\pmdefines{prime twin}
\pmdefines{prime quadruplet}
\endmetadata
\usepackage{amssymb}
\usepackage{amsmath}
\usepackage{amsfonts}
\usepackage{cnl}
\usepackage{xcolor}
%\usepackage{graphicx}
%%%%\usepackage{xypic}

%- DOCUMENT

\begin{document}
\parskip =\baselineskip
	
% Local Defs 
\def\natdiv#1#2{{#1}\mathrel{|}{#2}}
	
\begin{cnl}
		
\Cnlinput{../Tex2CNL/package/cnlinit}
		
\bigskip
		
%-% BEGIN

\lsection{Interprime}
		
In this section, let $d,\ m,\ p,\ q,\ n$ be integers.
		
\dfn{ We say that $d$ is \df{positive } iff $d > 0$. }
		
\dfn{ We say that $d$ \df{divides} $m$ iff $d\ne 0$ and there
 exists an integer $r$ such that $m= d\*r$. }
	
\dfn{ We write  $\natdiv{d}{m}$ iff $d$ divides $m$. }

\dfn{ We say  $d$ is a \df{divisor} of $m$ iff $d$ divides $m$. }

\dfn{ Assume that $p$ is greater than $1$.  Then we say that $p$
 is a \df{prime} iff each positive divisor of $p$ is equal to $1$
 or equal to $p$.}

\dfn{ Assume that $p,\ q$ are primes. Then we say that $p,\ q$ 
are \df{consecutive primes} iff there exists no prime $r$ such 
that $r>p$ and $r<q$. }

\dfn{ We say $d$ is an \df{odd number} iff there exists an 
integer $m$ such that $d=2\*m+1$. }

\dfn{ We say $d$ is an \df{even number} iff there exists an 
integer $m$ such that $d=2\*m$. }

\dfn{ We say $n$ is an \df{interprime} between $p$ and $q$ iff
 $p,\q$ are odd consecutive primes and $n = (p + q)/2$. }

\deflabel {prime twin}
We say $p,\ q$ form a \df{prime twin} iff $p,\ q$ are primes 
and (p=q-2$ or $p=q+2$).
\end{definition}

\dfn{ We say $p$ is a \df{twin prime} iff there exists $q$ such 
that $p,\ q$ form a prime twin. }

\begin{theorem}[Prop1]
If $p,\ q$ form a prime twin then the interprime between 
$p$ and $q$ is even. 
\end{theorem}

\dfn{ We say $p$ and $q$ and $r$ and $s$ form a \df{prime quadruplet} 
iff $p=q-2$ and $q=r-4$ and $r=s-2$. }

\begin{theorem}[Prop2]
If $(p,\ q,\ r,\ s)$ form a prime quadruplet then the 
interprime $n$ between $q$ and $r$ is odd and $\natdiv{5}{n}$.
\end{theorem}

\begin{remark}
Given two consecutive odd primes, the $i$th prime $p_i$ and
 the next one, $p_{i + 1}$, an {\em interprime} $n$ is the
 arithmetic mean of the two: $$n = \frac{p_i + p_{i + 1}}{2}$$ 
Thus, $n - p_i = p_{i + 1} - n$, so alternatively $$n = p_i + 
\frac{p_{i + 1} - p_i}{2} = p_{i + 1} - \frac{p_{i + 1} 
- p_i}{2}.$$ For example, given the 269th and 270th primes, 
1723 and 1733, the interprime is 1728, and indeed 
$1728 - 1723 = 1733 - 1728 = 5$. Interprimes themselves 
are of course always composite, though not always even. 
An interprime between a twin prime will always be even, while 
an interprime between the second (ending in 3 in base 10) 
and third (ending in 7 in base 10) member of a prime quadruplet
 will always be odd and be divisible by 5.

The first few interprimes are 4, 6, 9, 12, 15, 18, 21, 26, 30, 
34, 39, 42, 45, 50, 56, 60, 64, 69, 72, 76, 81, 86, 93, 99, etc., 
listed in A024675 of Sloane's OEIS.
\end{remark}

\end{cnl}




\end{document}
