\documentclass[12pt]{article}

% \usepackage[utf8]{inputenc}
\usepackage{url}

\title{Proposal for the 2021 ICERM Workshop on \\ Artificial Intelligence and Theorem Proving }

\author{Thomas C. Hales \and Cezary Kaliszyk \and Stephan Schulz \and Josef Urban}
% \author{Thomas C. Hales\\
% \affiliation{University of Pittsburgh}\\
% \and
% Josef Urban\\
% \affiliation{Radboud University Nijmegen}\\
% \affiliation{The Netherlands}\\
% }

%\titlerunning{Machine Learning for Large-Scale Theorem Proving}
%\authorrunning{Hales, Urban}

\begin{document}

\maketitle


% A description of the program area/theme (written with a general mathematical audience in mind),
% A list of organizers (normally around 4-7),
% A main contact (chair) of organizing committee,
% A discussion of the experimental and computational aspects of the program,
% The expected benefits of the proposed program,
% Plans for ensuring the participation of underrepresented groups (organizers are expected to work with ICERM directors on diversity issues).



\subsection*{Context}

Large-scale semantic processing and strong computer assistance of
mathematics and science is our inevitable future. New combinations of
AI and reasoning methods and tools deployed over large mathematical
and scientific corpora will be instrumental to this task.  The AITP
event is the forum for discussing how to get there as soon as
possible, and the force driving the progress towards that. AITP has
been organized yearly since 2016 in Europe growing to almost 80
participants in 2019. There is a consensus to organize AITP'21 in the
US.

The AITP topics include:
\begin{itemize}
\item AI and big-data methods in theorem proving and mathematics
\item Collaboration between automated and interactive theorem proving, in particular their AI/ML aspects
\item Common-sense reasoning and reasoning in science
\item Alignment and joint processing of formal, semi-formal, and informal libraries, Formal Abstracts
\item Methods for large-scale computer understanding of mathematics and science
\item Combinations of linguistic/learning-based and semantic/reasoning methods 
\item Formal verification of AI and machine learning algorithms, explainable AI 
\end{itemize}

\subsection*{Sessions, Speakers and Participants}

There will be several focused sessions on AI for ATP, ITP and
mathematics, modern AI and big-data methods, and several sessions with
contributed talks.  See the past AITP
programs\footnote{\url{http://aitp-conference.org/2019/},
  http://aitp-conference.org/2018/} for examples.  The focused
sessions will be based on invited talks by a selection of invited
speakers and discussion oriented. Typically, there are 10-20 confirmed
established researchers giving 6-10 invited talks. There will be 25-35
contributed talks (20-30 minutes) selected by the AITP program
committee based on extended abstracts of the proposed talks. An open
call for the abstracts will be published.


\subsection*{Preferred dates}

May/June 2021. 


\subsection*{Organiser details}

\begin{itemize}
%\item Pascal Fontaine (University of Lorraine \& INRIA, France,
%  \url{Pascal.Fontaine@loria.fr})
\item Thomas C. Hales,
University of Pittsburgh
  \url{hales@pitt.edu}
\item Cezary Kaliszyk
  University of Innsbruck
\item John Lafferty, ...
\item Stephan Schulz (Baden Wuerttemberg Cooperative State University, \url{schulz@eprover.org})
\item Josef Urban (main contact),
  Czech Technical University in Prague, Czech Republic\\
  \url{josef.urban@gmail.com}
\end{itemize}

\subsection*{Experimental and computational aspects}
todo

\subsection*{Expected benefits}
todo

\subsubsection*{Diversity}

The event participation will be based on merit without any
discrimination. The contributed talks will be selected by a
peer-review done by a program committee composed of established
researchers in the field across gender, race, etc.



\bibliographystyle{abbrv}
\bibliography{biblio}

\end{document}




